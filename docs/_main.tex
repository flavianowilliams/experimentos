% Options for packages loaded elsewhere
\PassOptionsToPackage{unicode}{hyperref}
\PassOptionsToPackage{hyphens}{url}
\documentclass[
]{book}
\usepackage{xcolor}
\usepackage{amsmath,amssymb}
\setcounter{secnumdepth}{5}
\usepackage{iftex}
\ifPDFTeX
  \usepackage[T1]{fontenc}
  \usepackage[utf8]{inputenc}
  \usepackage{textcomp} % provide euro and other symbols
\else % if luatex or xetex
  \usepackage{unicode-math} % this also loads fontspec
  \defaultfontfeatures{Scale=MatchLowercase}
  \defaultfontfeatures[\rmfamily]{Ligatures=TeX,Scale=1}
\fi
\usepackage{lmodern}
\ifPDFTeX\else
  % xetex/luatex font selection
\fi
% Use upquote if available, for straight quotes in verbatim environments
\IfFileExists{upquote.sty}{\usepackage{upquote}}{}
\IfFileExists{microtype.sty}{% use microtype if available
  \usepackage[]{microtype}
  \UseMicrotypeSet[protrusion]{basicmath} % disable protrusion for tt fonts
}{}
\makeatletter
\@ifundefined{KOMAClassName}{% if non-KOMA class
  \IfFileExists{parskip.sty}{%
    \usepackage{parskip}
  }{% else
    \setlength{\parindent}{0pt}
    \setlength{\parskip}{6pt plus 2pt minus 1pt}}
}{% if KOMA class
  \KOMAoptions{parskip=half}}
\makeatother
\usepackage{longtable,booktabs,array}
\usepackage{calc} % for calculating minipage widths
% Correct order of tables after \paragraph or \subparagraph
\usepackage{etoolbox}
\makeatletter
\patchcmd\longtable{\par}{\if@noskipsec\mbox{}\fi\par}{}{}
\makeatother
% Allow footnotes in longtable head/foot
\IfFileExists{footnotehyper.sty}{\usepackage{footnotehyper}}{\usepackage{footnote}}
\makesavenoteenv{longtable}
\usepackage{graphicx}
\makeatletter
\newsavebox\pandoc@box
\newcommand*\pandocbounded[1]{% scales image to fit in text height/width
  \sbox\pandoc@box{#1}%
  \Gscale@div\@tempa{\textheight}{\dimexpr\ht\pandoc@box+\dp\pandoc@box\relax}%
  \Gscale@div\@tempb{\linewidth}{\wd\pandoc@box}%
  \ifdim\@tempb\p@<\@tempa\p@\let\@tempa\@tempb\fi% select the smaller of both
  \ifdim\@tempa\p@<\p@\scalebox{\@tempa}{\usebox\pandoc@box}%
  \else\usebox{\pandoc@box}%
  \fi%
}
% Set default figure placement to htbp
\def\fps@figure{htbp}
\makeatother
\setlength{\emergencystretch}{3em} % prevent overfull lines
\providecommand{\tightlist}{%
  \setlength{\itemsep}{0pt}\setlength{\parskip}{0pt}}
\usepackage[]{natbib}
\bibliographystyle{plainnat}
\usepackage{booktabs}
\usepackage[T1]{fontenc}
\usepackage[portuguese]{babel}
\usepackage{amsmath}
\usepackage[a4paper, total={6in, 8in}]{geometry}
\usepackage{bookmark}
\IfFileExists{xurl.sty}{\usepackage{xurl}}{} % add URL line breaks if available
\urlstyle{same}
\hypersetup{
  pdftitle={Práticas experimentais para cursos de Física do ensino médio},
  pdfauthor={Andressa Sabrina Rodrigues Ferreira (discente); Aron Gabriel Medeiros de Lima (discente); Antônio Denizar Marques Patchek Franco (discente); Felipe Miguel Patchek Franco (discente); Joivana de Fátima Rodrigues Lau (discente); Sabrina Vitoria Wersel (discente); Vinicius Alady Jankovski (discente); Flaviano Williams Fernandes (docente)},
  hidelinks,
  pdfcreator={LaTeX via pandoc}}

\title{Práticas experimentais para cursos de Física do ensino médio}
\author{Andressa Sabrina Rodrigues Ferreira (discente) \and Aron Gabriel Medeiros de Lima (discente) \and Antônio Denizar Marques Patchek Franco (discente) \and Felipe Miguel Patchek Franco (discente) \and Joivana de Fátima Rodrigues Lau (discente) \and Sabrina Vitoria Wersel (discente) \and Vinicius Alady Jankovski (discente) \and Flaviano Williams Fernandes (docente)}
\date{04-12-2025}

\begin{document}
\maketitle

{
\setcounter{tocdepth}{1}
\tableofcontents
}
\chapter{Introdução}\label{introduuxe7uxe3o}

Este trabalho visar apresentar atividades práticas para aulas demonstrativas de física para alunos do ensino médio

\chapter{Elevador Hidráulico e o Princípio de Pascal}\label{elevador-hidruxe1ulico-e-o-princuxedpio-de-pascal}

\section{Autora: Andressa Sabrina Rodrigues Ferreira}\label{autora-andressa-sabrina-rodrigues-ferreira}

\section{Introdução}\label{introduuxe7uxe3o-1}

O estudo do comportamento dos fluidos e da forma como a pressão é transmitida em seu interior é fundamental para compreender diversas tecnologias presentes no cotidiano. Entre os conceitos centrais da hidrostática, destaca-se o Princípio de Pascal, formulado pelo cientista francês Blaise Pascal (1623-1662), que estabeleceu que ``a pressão exercida em um fluido confinado é transmitida integralmente e igualmente em todas as direções''
\citep{Hidalgo2021}. Esse princípio tornou-se a base para o desenvolvimento de máquinas hidráulicas, revolucionando a engenharia e a mecânica modernas.
O elevador hidráulico é um dos exemplos mais significativos da aplicação desse princípio. Presente em edifícios, indústrias e oficinas, ele funciona por meio da transmissão da pressão de um fluido geralmente óleo ou água que, quando comprimido em um pistão menor, é distribuído uniformemente e eleva um pistão maior, permitindo a elevação de cargas pesadas com a aplicação de uma força relativamente pequena. Essa característica de multiplicação da força torna os elevadores hidráulicos eficientes, seguros e economicamente viáveis para o transporte vertical de cargas e pessoas.
Compreender a relação entre o Princípio de Pascal e o funcionamento do elevador hidráulico possibilita conectar conceitos teóricos da Física a aplicações práticas e tecnológicas que facilitam a vida cotidiana. Além disso, o estudo experimental desses sistemas contribui para o desenvolvimento do raciocínio científico e para a valorização das descobertas que impulsionaram o progresso industrial e urbano. Dessa forma, a análise do elevador hidráulico e de sua base física demonstra a importância da ciência na resolução de problemas práticos e na construção de soluções inovadoras para a sociedade contemporânea.

\section{Objetivo}\label{objetivo}

Demonstrar, por meio de um experimento simples, o funcionamento de um elevador hidráulico, evidenciando a transmissão de pressão em fluidos e relacionando-a às aplicações no cotidiano.

\section{Parte experimental}\label{parte-experimental}

\subsection{Materiais utilizados}\label{materiais-utilizados}

\begin{itemize}
\tightlist
\item
  2 seringas sem agulha (60 mL e 40mL);
\item
  1 mangueira de silicone transparente;
\item
  Água colorida (opcional);
\item
  Madeira para fixar a seringa;
\item
  Pesos (ex.: caixa de MDF, brinquedos, etc.);
\item
  Pregos 12 Polegadas para fixação das madeiras;
\item
  Prego 17(para fazer os furos na madeira e fixará a seringa)
\item
  Arame (para fixar a seringa)
\item
  Papel e caneta para anotações;
\item
  Madeiras recicladas (refugo, cavacos de caixaria de portas);
\item
  Martelo;
\item
  Régua;
\item
  Palitos de churrasco reaproveitados de outro experimento;
\item
  Fita dupla face.
\item
  2 seringas sem agulha (60 mL e 40mL);
\item
  1 mangueira de silicone transparente;
\item
  Água colorida (opcional);
\item
  Madeira para fixar a seringa;
\item
  Pesos (ex.: caixa de MDF, brinquedos, etc.);
\item
  Pregos 12 Polegadas para fixação das madeiras;
\item
  Prego 17(para fazer os furos na madeira e fixará a seringa)
\item
  Arame (para fixar a seringa)
\item
  Papel e caneta para anotações;
\item
  Madeiras recicladas (refugo, cavacos de caixaria de portas);
\item
  Martelo;
\item
  Régua;
\item
  Palitos de churrasco reaproveitados de outro experimento;
\item
  Fita dupla face.
\end{itemize}

\subsection{Procedimento}\label{procedimento}

\begin{figure}

{\centering \includegraphics[width=0.8\linewidth]{img/01/figura-1a} 

}

\caption{Sistema sem aplicação da pressão.}\label{fig:figura-1a-1}
\end{figure}
\begin{figure}

{\centering \includegraphics[width=0.8\linewidth]{img/01/figura-1b} 

}

\caption{Sistema hidráulico em funcionamento, compressão e peso.}\label{fig:figura-1a-2}
\end{figure}

\subsubsection{Preparação dos materiais}\label{preparauxe7uxe3o-dos-materiais}

O experimento foi construído utilizando uma madeira de MDF com dimensões de 50 cm x 25 cm, que serviu como base da estrutura. Foram aproveitadas ripinhas de madeira cortadas com uma serrinha em 23 cm de comprimento, retiradas de restos de portas e móveis. Para a montagem, utilizaram-se pregos número 12, responsáveis por unir as partes móveis e permitir a articulação entre as peças. Também foram utilizados palitos de churrasco, duas seringas (uma de 60 mL e outra de 40 mL de menor volume), mangueira plástica e água e corante alimentar para facilitar a visualização do fluido.

\subsubsection{Montagem da estrutura}\label{montagem-da-estrutura}

As ripinhas foram lixadas para remover farpas e uniformizar as bordas. Em seguida, cada uma recebeu três furos, permitindo a articulação do mecanismo. As ripas foram fixadas de forma simétrica na base de MDF, mantendo uma distância de 14 cm entre elas, o que garantiu estabilidade e alinhamento adequado.

\subsubsection{Instalação do Sistema Hidráulico}\label{instalauxe7uxe3o-do-sistema-hidruxe1ulico}

A seringa maior (60mL) foi fixada horizontalmente na estrutura de madeira, representando o pistão principal do elevador. A seringa menor foi conectada à primeira através da mangueira plástica, formando o circuito hidráulico. Ambas foram preenchidas com água colorida, certificando-se de eliminar todas as bolhas de ar, para que a transmissão de pressão ocorresse corretamente.

\subsubsection{Teste de Funcionamento}\label{teste-de-funcionamento}

O êmbolo da seringa menor foi pressionado lentamente, observando-se o deslocamento do fluido e a consequente elevação do êmbolo da seringa maior, comprovando a transmissão de pressão entre os pistões. Ao liberar o êmbolo, o sistema retornava à posição inicial.O êmbolo da seringa menor foi pressionado lentamente, observando-se o deslocamento do fluido e a consequente elevação do êmbolo da seringa maior, comprovando a transmissão de pressão entre os pistões. Ao liberar o êmbolo, o sistema retornava à posição inicial.

\subsubsection{Demonstração da Capacidade de Elevação}\label{demonstrauxe7uxe3o-da-capacidade-de-elevauxe7uxe3o}

Para evidenciar o funcionamento do elevador, foram fixados palitos de churrasco com fita dupla face na extremidade da seringa maior, servindo de suporte para pequenos pesos. Também foi utilizada uma caixinha de MDF para simular a carga a ser erguida, permitindo observar a eficiência do sistema.

\textbf{Nota} O uso de água colorida ajuda a visualizar o deslocamento do fluido.

\section{Questionário}\label{questionuxe1rio}

\begin{enumerate}
\def\labelenumi{\arabic{enumi}.}
\item
  Explique, com suas palavras, o Princípio de Pascal.
\item
  Por que é essencial que o fluido seja incompressível no funcionamento do sistema hidráulico?
\item
  Cite três exemplos do uso do Princípio de Pascal no cotidiano.
\item
  Qual a função da diferença de tamanho entre as seringas no experimento?
\item
  Que vantagens o elevador hidráulico oferece em termos de esforço humano?
\item
  O que você percebeu ao pressionar a seringa menor? Como isso demonstra o Princípio de Pascal?
\end{enumerate}

\chapter{Construção e Utilização de um Calorímetro Caseiro com Caixa de Leite}\label{construuxe7uxe3o-e-utilizauxe7uxe3o-de-um-caloruxedmetro-caseiro-com-caixa-de-leite}

\section{Autor: Aron Gabriel Medeiros de Lima}\label{autor-aron-gabriel-medeiros-de-lima}

\section{Introdução}\label{introduuxe7uxe3o-2}

O calorímetro é um equipamento fundamental no estudo da Calorimetria, a área da Física que estuda a troca de calor entre corpos e as mudanças de estado físico. Um calorímetro ideal é projetado para ser um sistema isolado, onde as trocas de calor ocorrem apenas entre os componentes internos (substâncias) e o próprio aparelho, minimizando a perda de calor para o ambiente externo.

A construção de um calorímetro caseiro, utilizando materiais de baixo custo e fácil acesso, como a caixa de leite (embalagem cartonada longa vida), tubo de ensaio e termômetro, permite a realização de experimentos práticos para a compreensão dos conceitos de calor específico, capacidade térmica e energia liberada em reações ou na queima de substâncias.

\section{Objetivo}\label{objetivo-1}

Construir um calorímetro simples utilizando materiais reciclados e de baixo custo para realizar medições aproximadas de variações de temperatura e, consequentemente, do calor envolvido em processos.

\section{Materiais necessários}\label{materiais-necessuxe1rios}

\begin{itemize}
\tightlist
\item
  1 Caixa de leite (embalagem cartonada longa vida) vazia e limpa.
\item
  1 Tubo de ensaio.
\item
  1 Termômetro (preferencialmente de precisão, mas um termômetro culinário ou clínico pode ser adaptado).
\item
  Água.
\item
  Tesoura ou estilete.
\item
  Fita adesiva ou cola.
\item
  Suporte para o tubo de ensaio (opcional, para mantê-lo suspenso ou fixo).
\item
  Fonte de calor (ex: vela, lamparina a álcool ou alimento a ser queimado).
\item
  Proveta ou recipiente medidor (para medir a água).
\end{itemize}

\section{Montagem do calorímetro}\label{montagem-do-caloruxedmetro}

O calorímetro caseiro, baseado na estrutura de uma caixa de leite, busca aproveitar as camadas isolantes (geralmente papelão, alumínio e polietileno) da embalagem cartonada para reduzir a troca de calor com o ambiente.

\begin{enumerate}
\def\labelenumi{\arabic{enumi}.}
\item
  Preparação da Caixa de Leite (Revestimento/Isolamento)

  \begin{enumerate}
  \def\labelenumii{\alph{enumii})}
  \item
    Corte a parte superior da caixa (a que costuma ser usada para abrir) na horizontal e descarte-a. A caixa deve formar um recipiente aberto em uma das extremidades.
  \item
    Opcional: Para melhorar o isolamento, forre o interior da caixa com papel alumínio (com o lado brilhante para dentro) ou isopor fino.
  \item
    Abertura na parte da frente: Faça uma abertura de tamanho razoável na parte lateral ou frontal da caixa, perto da base, que permitirá a colocação e ignição da fonte de calor (ex: o alimento a ser queimado).
  \end{enumerate}
\item
  Posicionamento do Tubo de Ensaio:

  \begin{enumerate}
  \def\labelenumii{\alph{enumii})}
  \item
    Meça uma quantidade conhecida de água (por exemplo, 10 mL ou 20 mL, dependendo do tamanho do tubo de ensaio) e coloque-a dentro do tubo de ensaio.
  \item
    Introduza o termômetro dentro do tubo de ensaio, imerso na água. Atenção: O termômetro não deve tocar o fundo ou as paredes do tubo.
  \item
    Se necessário, crie uma ``tampa'' improvisada (por exemplo, com papelão ou pedaço da própria caixa) para o tubo de ensaio, com um pequeno furo para o termômetro, para reduzir a perda de calor da água para o ar.
  \end{enumerate}
\end{enumerate}

\section{Aplicação (Exemplo: Medição de Energia de um Alimento)}\label{aplicauxe7uxe3o-exemplo-mediuxe7uxe3o-de-energia-de-um-alimento}

Este tipo de calorímetro é frequentemente usado para estimar a energia calórica (calorias) liberada na queima de um alimento, como um amendoimEste tipo de calorímetro é frequentemente usado para estimar a energia calórica (calorias) liberada na queima de um alimento, como um amendoim

\begin{enumerate}
\def\labelenumi{\arabic{enumi}.}
\item
  Medir as Temperaturas:

  \begin{enumerate}
  \def\labelenumii{\alph{enumii})}
  \tightlist
  \item
    Anote a temperatura inicial (Tinicial) da água no tubo de ensaio.
  \end{enumerate}
\item
  Queima do Alimento:

  \begin{enumerate}
  \def\labelenumii{\alph{enumii})}
  \item
    Coloque o alimento (ex: um amendoim preso em um arame) em um suporte de ignição (pode ser um pedaço de rolha ou isopor com um clipe esticado) e posicione-o na abertura da frente do calorímetro.Coloque o alimento (ex: um amendoim preso em um arame) em um suporte de ignição (pode ser um pedaço de rolha ou isopor com um clipe esticado) e posicione-o na abertura da frente do calorímetro.
  \item
    Acenda o alimento com um palito de fósforo ou vela, e rapidamente introduza-o na abertura frontal, sob a base do tubo de ensaio com água.
  \item
    Deixe o alimento queimar completamente. O calor liberado aquecerá a água.
  \end{enumerate}
\item
  Medir a Temperatura Final:

  \begin{enumerate}
  \def\labelenumii{\alph{enumii})}
  \item
    Após a queima total do alimento e a estabilização da temperatura, anote a temperatura final (Tfinal) da água.
  \item
    Calcule a variação de temperatura : \(\Delta T = T_{final}-T_{inicial}\).
  \end{enumerate}
\end{enumerate}

\section{Cálculos envolvidos}\label{cuxe1lculos-envolvidos}

O calor (Q) recebido pela água e, idealmente, pelo calorímetro pode ser calculado pela equação do \textbf{Calor Sensível} (chamada de equação fundamental da calorimetria, ou ``Que MaCeTe''):

\begin{align}
Q & = mc\Delta T,
\end{align}

onde

\begin{itemize}
\tightlist
\item
  Q é o calor recebido pela água (em calorias, cal).
\item
  m é a massa da água (em gramas, g).
\item
  c é o calor específico da água (1 cal/g).
\item
  \(\Delta T\) é a variação de temperatura (graus Celsius).
\end{itemize}

Considerações:

\begin{itemize}
\tightlist
\item
  Em um calorímetro caseiro, o valor de Q calculado é apenas uma estimativa da energia liberada, pois uma quantidade significativa de calor é perdida para o ar, para a caixa de leite e para o tubo de ensaio (que também absorvem calor, mas que não foram incluídos no cálculo simples).
\item
  Em experimentos mais detalhados, é preciso considerar a Capacidade Térmica do Calorímetro (cal), ou seja, a energia que o próprio calorímetro (caixa, tubo, etc.) absorve. A fórmula completa seria: Qtotal = Qágua + Qcalorímetro.
\item
  A energia liberada pelo alimento (Ealimento) é igual ao calor total absorvido: E alimento = Q total.
\end{itemize}

\section{Conclusão}\label{conclusuxe3o}

O calorímetro caseiro com caixa de leite e tubo de ensaio é uma ferramenta didática eficiente que demonstra os princípios básicos da calorimetria. Embora apresente limitações de precisão devido às perdas de calor para o ambiente, ele permite aos estudantes visualizar a transferência de energia e aplicar as fórmulas fundamentais de cálculo de calor, tornando o estudo de termodinâmica mais concreto e acessível.

\section{Questionário}\label{questionuxe1rio-1}

\subsection{Parte I: Questões de Múltipla Escolha e Conceituais}\label{parte-i-questuxf5es-de-muxfaltipla-escolha-e-conceituais}

\subsubsection{Questão}\label{questuxe3o}

Qual é a principal função da caixa de leite (embalagem cartonada longa vida) no calorímetro caseiro, do ponto de vista da termodinâmica?

A. Servir como fonte de calor inicial.

B. Atuar como um agente misturador para a água.

C. Maximizar a transferência de calor para o ambiente.

D. Reduzir a perda de calor por meio do isolamento térmico.

E. Reagir químicamente com o tubo de ensaio, estabilizando a temperatura.

\subsubsection{Questão}\label{questuxe3o-1}

O tubo de ensaio, que contém a água e o termômetro, é o receptor direto da energia liberada pela fonte de calor (ex: um amendoim queimando). Qual fenômeno físico explica o aquecimento da água nessa montagem?

\begin{enumerate}
\def\labelenumi{\Alph{enumi})}
\item
  Condução, pois o calor se propaga através das paredes do tubo de ensaio para a água, e radiação, proveniente da chama.
\item
  Apenas Convecção, pois a água circula dentro do tubo.
\item
  Apenas Radiação, pois o calor da chama atinge o tubo sem contato direto.
\item
  Condução, Radiação e Convecção (na água).
\item
  Apenas o trabalho realizado pela chama, que aumenta a energia interna da água.
\end{enumerate}

\subsubsection{Questão}\label{questuxe3o-2}

Na fórmula fundamental da calorimetria (Q=m⋅c⋅ΔT), o termo ΔT representa a variação de temperatura. Se 20 mL de água (assumindo densidade de 1 g/mL) sofrem um aumento de temperatura de 15g∘C durante um experimento, e o calor específico da água é 1 cal/g∘C, qual é o valor de calor mínimo absorvido (Qágua) pela água?

A. 35 cal

B. 20 cal

C. 300 cal

D. 150 cal

E. 200 cal

\subsubsection{Questão}\label{questuxe3o-3}

A principal limitação do calorímetro caseiro (feito com caixa de leite, tubo de ensaio e termômetro) para medir a energia de um alimento, em comparação com um calorímetro de bomba de nível laboratorial, é que o modelo caseiro:

A. Utiliza água em vez de óleo como fluido térmico.

B. Permite grande perda de calor para o ambiente (sistema não ideal).

C. Não permite a medição da massa inicial do alimento.

D. Possui capacidade térmica nula, distorcendo o resultado.

E. Não utiliza a unidade caloria (cal) em seus cálculos.

\subsection{Parte II: Questões Dissertativas e de Análise}\label{parte-ii-questuxf5es-dissertativas-e-de-anuxe1lise}

\subsubsection{Questão}\label{questuxe3o-4}

Defina o conceito de Capacidade Térmica (C) de um corpo e explique por que é crucial desconsiderar (ou minimizá-la) a Capacidade Térmica do calorímetro caseiro (do tubo de ensaio e da caixa) para que os resultados do experimento sejam mais próximos do calor real liberado pela fonte.

\subsubsection{Questão}\label{questuxe3o-5}

Um estudante mede a massa de 20 g de amendoim antes da queima. A massa final residual após a queima é de 19,5 g. O calorímetro caseiro contendo 50 g de água sofre uma variação de temperatura (ΔT) de 10∘C. Se a água absorveu 500 cal (Q=50 g⋅1 cal/g∘C⋅10∘C), calcule a quantidade de energia (em cal/g) liberada pelo amendoim considerando apenas o calor absorvido pela água.

\subsubsection{Questão}\label{questuxe3o-6}

O calorímetro ideal é um sistema isolado. No calorímetro caseiro, a perda de calor ocorre principalmente por dois mecanismos (além da condução para a estrutura). Quais são esses mecanismos e como a montagem com a caixa de leite e o tubo de ensaio busca atenuar a perda de calor do tubo para o ambiente?

\subsubsection{Questão}\label{questuxe3o-7}

Considere que, em um experimento mais preciso, a Capacidade Térmica do tubo de ensaio (Ctubo) e da caixa de leite (Ccaixa) em conjunto é de 20 cal/∘C. Se a variação de temperatura foi de 10∘C e o calor absorvido pela água foi 500 cal (conforme a Questão 6), calcule o calor total (Qtotal) liberado pela fonte de calor neste sistema não ideal.

\subsubsection{Questão}\label{questuxe3o-8}

Explique, usando os conceitos de calorimetria, por que o termômetro não deve tocar o fundo do tubo de ensaio durante a medição.

\subsubsection{Questão}\label{questuxe3o-9}

Qual seria o impacto no valor do calor específico calculado para uma amostra (em um experimento de troca de calor) se o calorímetro caseiro fosse montado com uma caixa de leite sem o forro de alumínio interno (que é a camada reflexiva da embalagem)? Justifique sua resposta.

\section{Gabarito e resolução detalhada}\label{gabarito-e-resoluuxe7uxe3o-detalhada}

\subsection{Parte I: Respostas}\label{parte-i-respostas}

\begin{longtable}[]{@{}cc@{}}
\toprule\noalign{}
Questão & Resposta correta \\
\midrule\noalign{}
\endhead
\bottomrule\noalign{}
\endlastfoot
1 & D \\
2 & D \\
3 & C \\
4 & B \\
\end{longtable}

\subsection{Parte II: Questões Dissertativas e de Análise}\label{parte-ii-questuxf5es-dissertativas-e-de-anuxe1lise-1}

\subsubsection{Questão 5: Capacidade Térmica (C)}\label{questuxe3o-5-capacidade-tuxe9rmica-c}

A Capacidade Térmica é o calor necessário para variar a temperatura de um corpo em um grau. É crucial minimizá-la porque o calor absorvido pelo calorímetro (tubo e caixa) não é contabilizado, levando a uma subestimação do calor total liberado pela fonte se essa capacidade for alta.

\subsubsection{Questão 6: Energia por Grama}\label{questuxe3o-6-energia-por-grama}

A massa queimada é de 0,5 g (20 g−19,5 g). Se a água absorveu 500 calorias, a energia liberada é: 500 cal/0,5 g=1000 cal/g.
Questão 7: Mecanismos de Perda e Atenuação
Os principais mecanismos de perda são Convecção (pelo ar) e Radiação (ondas de calor). A caixa de leite, com sua camada de alumínio, ajuda a refletir a radiação de volta para dentro do sistema e a conter o ar, minimizando a perda por convecção.

\subsubsection{Questão 8: Cálculo do Calor Total}\label{questuxe3o-8-cuxe1lculo-do-calor-total}

O calor total é a soma do calor da água e do calor absorvido pelo calorímetro.
Calor do Calorímetro: 20 cal x 100 = 200 cal
Calor Total: 500 cal + 200 cal = 700 calorias

\subsubsection{Questão 9: Posição do Termômetro}\label{questuxe3o-9-posiuxe7uxe3o-do-termuxf4metro}

O termômetro deve medir a temperatura da água, e não a do tubo de ensaio. Se tocar o fundo, medirá a temperatura da parede sólida, que é mais quente e sobe mais rápido, resultando em leituras incorretas e superestimadas do calor absorvido.

\subsubsection{Questão 10: Impacto da Ausência do Alumínio}\label{questuxe3o-10-impacto-da-ausuxeancia-do-alumuxednio}

O valor do calor específico calculado seria menor do que o valor real. Sem a camada de alumínio (reflexiva), haveria maior perda de calor por Radiação para o ambiente. Consequentemente, o calor absorvido pela água seria menor, subestimando o calor específico da amostra testada.

\chapter{FILTRO DE ÁGUA - Dois sistemas: Decantação e filtro}\label{filtro-de-uxe1gua---dois-sistemas-decantauxe7uxe3o-e-filtro}

\section{Autor: Antônio Denizar Marques Patchek Franco}\label{autor-antuxf4nio-denizar-marques-patchek-franco}

\section{Introdução}\label{introduuxe7uxe3o-3}

A água é um recurso essencial para a vida e sua qualidade está diretamente ligada à saúde e ao bem-estar da população. No entanto, nem sempre a água disponível para consumo encontra-se em condições seguras para todas as pessoas, sendo necessária a utilização de processos de tratamento e filtragem para garantir sua potabilidade. Compreender esses métodos é fundamental não apenas para a aprendizagem científica, mas também para o desenvolvimento da consciência ambiental e da responsabilidade social.

Podemos definir consciência ambiental como sendo a tendência de um indivíduo em se posicionar frente aos assuntos relativos ao meio ambiente de uma maneira a favor ou contra \citep{Gonalves-Dias2009}, nesse contexto significa a consciência do indivíduo entender que a água para consumo é limitada e que por isso precisamos evitar o desperdício.

Já a responsabilidade social está ligada a forma como o indivíduo age no sentido de preservar a água, onde ele tem clareza de que pequenas ações do dia a dia podem impactar no meio ambiente. Como exemplo podemos citar: evitar deixar a torneira aberta e descartar o lixo corretamente.

Portanto, aprender sobre a filtragem e o tratamento da água não é só um estudo científico, é também uma maneira de refletir sobre como podemos cuidar melhor do planeta e da sociedade. Dessa forma, cada um de nós pode ser parte da mudança para um mundo mais saudável e sustentável.

Diante disso, esse trabalho tem como objetivo apresentar a importância da água tratada para o consumo humano, destacando os diferentes métodos de separação e filtragem existentes, sendo eles físicos, químicos e biológicos. O método físico é aquele que retira as impurezas sólidas da água sem alterar sua composição química. Exemplo a filtração com areia e cascalho que quando a água passa por camadas de areia e pedras, as partículas ficam retidas, fazendo com que a água fique limpa. O método químico, usa reações químicas ou substâncias para remover possíveis contaminações. Como exemplo, podemos citar o uso de cloro e o uso de carvão ativado. Já o método biológico, utiliza seres vivos, principalmente micro-organismos para degradar poluentes. Exemplo: Biorremediação de filtros biológicos, muito usados em tratamento de esgoto.

\section{Objetivos}\label{objetivos}

\begin{itemize}
\tightlist
\item
  Apresentar a importância da água tratada para o consumo humano, destacando a sua relevância em nossas vidas e no ecossistema;
\item
  Adquirir conhecimentos sobre técnicas de tratamento caseiro de água.
\end{itemize}

\section{PARTE EXPERIMENTAL (fisica e quimica)}\label{parte-experimental-fisica-e-quimica}

Primeiramente é importante falar sobre os elementos químicos que formam a água, sua origem, sua importância e onde é encontrada. Também contextualizar as formas de filtragem conhecidas (físico, químico e biológico). Demonstrar aos alunos os métodos de filtragem de água: Filtragem por Decantação (neste método, demonstramos como separar partículas sólidas em suspensão, deixando a água em repouso até que os resíduos se depositem no fundo do recipiente) e Filtragem com Sistema Caseiro (montar um sistema de filtragem utilizando os seguintes materiais: Bomba de máquina de lavar roupa, Mangueiras transparentes, Recipientes transparentes e Elementos filtrantes: algodão, carvão ativado, areia fina e pedra brita. Esse sistema simula um filtro doméstico, mostrando como cada camada atua na retenção de impurezas da água). Os alunos podem ser divididos em grupos de 3 a 4 integrantes, onde cada grupo receberá materiais diversos, como: garrafas PET, areia, pedras, carvão, algodão, copos descartáveis, grama, madeiras, panos entre outros. O objetivo é que eles usem a criatividade para construir seu próprio filtro de água funcional, aplicando os conceitos aprendidos na parte teórica e nas demonstrações. Cada grupo terá um tempo estipulado para apresentar o modelo de filtro criado, bem como explicar os materiais utilizados e assim demonstrar a passagem da água pelo filtro e comentar o que aprenderam com a experiência.

\subsection{Materiais utilizados}\label{materiais-utilizados-1}

\begin{itemize}
\item
  Garrafas PET;
\item
  Areia;
\item
  Pedras;
\item
  Carvão;
\item
  Algodão;
\item
  Madeiras;
\item
  Bicos de câmara de pneu;
\item
  Mangueiras.
\end{itemize}

\subsection{Procedimento experimental}\label{procedimento-experimental}

Para a confecção desse filtro caseiro será utilizado os seguintes materiais: dois pedaços de madeira: 1 de aproximadamente 1,50 cm e outro de 0,50 cm, uma base de madeira (ou outro tipo de material), 4 garrafas pet, 4 parafusos para fixar a garrafa e 4 mangueira de aproximadamente 4 mm, com 50 cm, 6 bicos de câmera de pneu que serão usados para fixar as mangueiras nas garrafas. Modo de decantação: Aproximadamente de 8 a 10 cm de distância entre um litro e outro para que ocorra a decantação que é o processo onde os resíduos ficarão na parte baixa do filtro. São três sistemas de decantação, sendo o quarto pet é o filtro. O filtro será feito também existindo um sistema de decantação, ficando a parte de saída da água a uma altura de 08 a 10 cm, contendo em primeira camada aproximadamente 1kg de pedra, na segunda camada 1 kg de areia, na terceira camada carvão, na quarta camada filtro de café e se preferir colocar algodão na última saída de água. Ressaltando que todos esses materiais, exceto a madeira, terá que ser higienizado, areia, pedra e carvão, terão que ser limpos e secos. Os bicos de conexão entre a mangueira e outra precisam ser lavados e esterilizados.

\begin{figure}

{\centering \includegraphics[width=0.8\linewidth]{img/03/figura-3a} 

}

\caption{...}\label{fig:figura-3a}
\end{figure}

\section{Questionário}\label{questionuxe1rio-2}

\subsection{Qual é o principal objetivo de filtrar a água?}\label{qual-uxe9-o-principal-objetivo-de-filtrar-a-uxe1gua}

Resposta: O objetivo é tratar a água para o consumo humano.

\subsection{Quais são os três tipos de métodos de tratamento da água citados e como funcionam?}\label{quais-suxe3o-os-truxeas-tipos-de-muxe9todos-de-tratamento-da-uxe1gua-citados-e-como-funcionam}

Resposta:

\begin{itemize}
\item
  Físico: remove impurezas sólidas sem alterar a composição da água (ex: filtração com areia e cascalho).
\item
  Químico: utiliza substâncias ou reações químicas para eliminar contaminações (ex: uso de cloro e carvão ativado).
\item
  Biológico: emprega micro-organismos para degradar poluentes (ex: biorremediação em filtros biológicos).
\end{itemize}

\subsection{Como funciona o processo de decantação demonstrado no experimento?}\label{como-funciona-o-processo-de-decantauxe7uxe3o-demonstrado-no-experimento}

Resposta: A decantação ocorre quando a água é deixada em repouso, permitindo que as partículas sólidas mais pesadas se depositem no fundo do recipiente, separando-se da parte líquida.

\subsection{Quais materiais são usados na montagem do filtro caseiro?}\label{quais-materiais-suxe3o-usados-na-montagem-do-filtro-caseiro}

Resposta:
São utilizados garrafas PET, areia, pedras, carvão, algodão, madeiras, bicos de câmara de pneu e mangueiras. Esses materiais formam as camadas filtrantes e o sistema de passagem da água.

\subsection{Quais são as camadas do filtro?}\label{quais-suxe3o-as-camadas-do-filtro}

Resposta:
O filtro é formado por quatro camadas, sendo elas: pedra, areia, carvão e filtro de café e algodão.

\subsection{Qual a função da pedra no filtro?}\label{qual-a-funuxe7uxe3o-da-pedra-no-filtro}

Resposta:
A camada de pedras serve para reter partículas grandes;

\chapter{Experimento da Alavanca: Estudo das Leis da Física Aplicadas ao Princípio da Alavanca}\label{experimento-da-alavanca-estudo-das-leis-da-fuxedsica-aplicadas-ao-princuxedpio-da-alavanca}

\section{Autor: Felipe Miguel Patchek Franco}\label{autor-felipe-miguel-patchek-franco}

\section{Introdução}\label{introduuxe7uxe3o-4}

O experimento da alavanca visa explorar as leis fundamentais da física, especialmente o Princípio da Alavanca, que está relacionado ao conceito de momento de força. A alavanca é uma máquina simples que permite a ampliação de uma força, facilitando a realização de trabalho. Esse experimento tem como objetivo verificar a relação entre a força aplicada, a distância entre o ponto de apoio e o ponto de aplicação da força, e a equação que descreve o equilíbrio da alavanca.
O estudo da alavanca é um exemplo clássico de como os conceitos de física, como força, torque e equilíbrio, podem ser aplicados no dia a dia e em diversas máquinas e dispositivos. De forma simples, ao mover um ponto de apoio, podemos alterar a quantidade de força necessária para equilibrar objetos pesados, facilitando o trabalho humano.

\section{Objetivos}\label{objetivos-1}

\begin{itemize}
\item
  Compreender o conceito de alavanca e sua aplicação prática.
\item
  Estudar a relação entre força, distância e ponto de apoio, conforme a Lei da Alavanca.
\item
  Determinar a condição de equilíbrio de uma alavanca.
\item
  Verificar a fórmula de equilíbrio de uma alavanca (fórmula de momentos).
\end{itemize}

\section{Parte experimental}\label{parte-experimental-1}

\subsection{Materiais utilizados}\label{materiais-utilizados-2}

\begin{itemize}
\item
  1 base de papelão firme (para servir de suporte)
\item
  1 tira de papelão (aproximadamente 25--30 cm de comprimento)
\item
  1 pincel, lápis ou canudo rígido (para atuar como fulcro/ponto de apoio)
\item
  2 tampinhas de garrafa PET e tampinhas de tinta Guache (uma para cada extremidade da alavanca)
\item
  3 moedas (duas coladas em uma tampinha e uma na outra)
\item
  Fita adesiva transparente ou cola quente (para fixação)
\item
  Tesoura ou estilete (para cortar o papelão)
\item
  Régua e caneta (para medir e marcar as distâncias)
\item
  (Opcionais) massinha de modelar para ajuste de pesos ou equilíbrio e papéis EVA para decorar.
\end{itemize}

\subsection{Procedimento experimental}\label{procedimento-experimental-1}

\begin{enumerate}
\def\labelenumi{\arabic{enumi}.}
\tightlist
\item
  Preparação da base: Inicialmente, foi recortado um pedaço de \textbf{papelão} utilizando \textbf{tesoura}, que serviu como base de sustentação da montagem.
\end{enumerate}

\begin{figure}

{\centering \includegraphics[width=0.8\linewidth]{img/rascunho} 

}

\caption{...}\label{fig:figura-4a}
\end{figure}

\begin{enumerate}
\def\labelenumi{\arabic{enumi}.}
\setcounter{enumi}{1}
\tightlist
\item
  Montagem dos suportes: Em seguida, foram recortados e colados \textbf{dois suportes verticais de papelão} sobre a base, deixando entre eles um espaço para posicionar o pincel, que funcionou como o \textbf{fulcro (ponto de apoio)} da alavanca. Os suportes foram colados com fita adesiva e cola, e o pincel foi \textbf{perfurado} e encaixado nos suportes, de modo que pudesse girar livremente.
\end{enumerate}

\begin{figure}

{\centering \includegraphics[width=0.8\linewidth]{img/rascunho} 

}

\caption{...}\label{fig:figura-4b}
\end{figure}

\begin{enumerate}
\def\labelenumi{\arabic{enumi}.}
\setcounter{enumi}{2}
\tightlist
\item
  Construção da barra da alavanca: Foi cortado mais um pedaço retangular de papelão, que serviu como a barra da alavanca. Essa barra foi apoiada sobre o pincel, centralizada de modo que pudesse se equilibrar.
\end{enumerate}

\begin{figure}

{\centering \includegraphics[width=0.8\linewidth]{img/rascunho} 

}

\caption{...}\label{fig:figura-4c}
\end{figure}

\begin{enumerate}
\def\labelenumi{\arabic{enumi}.}
\setcounter{enumi}{3}
\tightlist
\item
  Fixação dos pesos: Em cada extremidade da barra de papelão, foi colada uma tampinha de garrafa PET. Em uma das tampinhas foram coladas duas moedas, representando um maior peso (força). Na outra tampinha foi colada apenas uma moeda, representando o menor peso.
\end{enumerate}

\begin{figure}

{\centering \includegraphics[width=0.8\linewidth]{img/rascunho} 

}

\caption{...}\label{fig:figura-4d}
\end{figure}

\begin{enumerate}
\def\labelenumi{\arabic{enumi}.}
\setcounter{enumi}{4}
\tightlist
\item
  Observação do funcionamento: A alavanca foi posicionada sobre o fulcro (pincel) e observou-se que o lado com duas moedas desceu, enquanto o lado com uma moeda subiu, demonstrando o desequilíbrio inicial. Ao mover o ponto de apoio ligeiramente em direção ao lado mais pesado, foi possível restabelecer o equilíbrio.
\end{enumerate}

\begin{figure}

{\centering \includegraphics[width=0.8\linewidth]{img/rascunho} 

}

\caption{...}\label{fig:figura-4e}
\end{figure}

\begin{enumerate}
\def\labelenumi{\arabic{enumi}.}
\setcounter{enumi}{5}
\tightlist
\item
  Conclusão visual: O experimento permitiu observar de forma simples e prática o princípio da alavanca, mostrando que o equilíbrio depende da relação entre força e distância em relação ao ponto de apoio. Observou-se que o peso indicada na balança variou quando o anel foi colocado, demonstrando que houve mudança na força exercida sobre ela.
\end{enumerate}

\begin{figure}

{\centering \includegraphics[width=0.8\linewidth]{img/rascunho} 

}

\caption{...}\label{fig:figura-4f}
\end{figure}

\section{Questionário}\label{questionuxe1rio-3}

\begin{enumerate}
\def\labelenumi{\arabic{enumi}.}
\tightlist
\item
  Qual é o objetivo do experimento realizado?
\end{enumerate}

O objetivo do experimento é demonstrar, de forma simples e prática, o princípio da alavanca, observando como o peso e a distância do ponto de apoio influenciam no equilíbrio e no movimento da barra.

\begin{enumerate}
\def\labelenumi{\arabic{enumi}.}
\setcounter{enumi}{1}
\tightlist
\item
  Que função o pincel desempenha na montagem da alavanca?
\end{enumerate}

O pincel atua como o fulcro ou ponto de apoio da alavanca. É em torno dele que a barra de papelão gira, permitindo o movimento de sobe e desce entre os dois lados.

\begin{enumerate}
\def\labelenumi{\arabic{enumi}.}
\setcounter{enumi}{2}
\tightlist
\item
  O que acontece quando há duas moedas de um lado e uma do outro?
\end{enumerate}

O lado com duas moedas fica mais pesado e desce, enquanto o lado com uma moeda fica mais leve e sobe. Isso ocorre porque a força peso é maior no lado das duas moedas.

\begin{enumerate}
\def\labelenumi{\arabic{enumi}.}
\setcounter{enumi}{3}
\tightlist
\item
  Como é possível equilibrar o sistema mesmo com pesos diferentes?
\end{enumerate}

Para equilibrar o sistema, basta mover o ponto de apoio (pincel) mais próximo do lado mais pesado. Assim, o braço de alavanca do lado mais leve fica maior e compensa a diferença de peso, equilibrando os torques.

\begin{enumerate}
\def\labelenumi{\arabic{enumi}.}
\setcounter{enumi}{4}
\tightlist
\item
  Escreva a expressão matemática que representa o equilíbrio de uma alavanca.
\end{enumerate}

A relação de equilíbrio é dada por

\begin{align}
F1 \cdot d1 & = F2 \cdot d2,\\
F_1\cdot d_1 & = F_2 \cdot d_2,\\
F1​\cdot d1& =F2\cdot d2
\end{align}

onde FFF é a força (ou peso) e ddd é a distância até o fulcro.

\begin{enumerate}
\def\labelenumi{\arabic{enumi}.}
\setcounter{enumi}{5}
\tightlist
\item
  Cite dois exemplos de alavancas que usamos no dia a dia.
\end{enumerate}

\begin{itemize}
\item
  Tesoura -- o fulcro é o parafuso no centro.
\item
  Alicate -- o fulcro é a articulação entre os cabos.
  Outros exemplos possíveis são gangorra, abridor de garrafa e chave de roda.
\end{itemize}

\begin{enumerate}
\def\labelenumi{\arabic{enumi}.}
\setcounter{enumi}{6}
\tightlist
\item
  Quais fatores podem causar erro experimental?
\end{enumerate}

\begin{itemize}
\item
  O atrito entre o pincel e o papelão.
\item
  Desalinhamento do fulcro.
\item
  Peso desigual das tampinhas ou do papelão.
\item
  Fixação incorreta das moedas. Esses fatores podem impedir que o equilíbrio ocorra exatamente como previsto.
\end{itemize}

\begin{enumerate}
\def\labelenumi{\arabic{enumi}.}
\setcounter{enumi}{7}
\tightlist
\item
  De que forma este experimento contribui para a aprendizagem prática da Física?
\end{enumerate}

O experimento ajuda a visualizar conceitos abstratos da Física, como torque, equilíbrio e força. Ele torna o aprendizado mais concreto e participativo, permitindo ao aluno experimentar e compreender o funcionamento de uma máquina simples.

\begin{enumerate}
\def\labelenumi{\arabic{enumi}.}
\setcounter{enumi}{8}
\tightlist
\item
  Qual a importância da curricularização da extensão em atividades como esta?
\end{enumerate}

A curricularização da extensão permite aproximar o ensino da realidade da comunidade, mostrando que a Física está presente em situações do dia a dia. Além disso, promove a criatividade, a sustentabilidade e o trabalho em equipe, valores importantes para a formação integral do estudante.

\chapter{Disco de Newton (Círculo Cromático de Newton)}\label{disco-de-newton-cuxedrculo-cromuxe1tico-de-newton}

\section{Autora: Joivana de Fátima Rodrigues Lau}\label{autora-joivana-de-fuxe1tima-rodrigues-lau}

\section{Introdução}\label{introduuxe7uxe3o-5}

O disco de Newton também conhecido como círculo cromático de Newton é um experimento amplamente reconhecido no campo da Física.A decomposição da luz observada no disco de Newton está intrinsecamente ligada ao experimento realizado por Isaac Newton com um prisma. Consiste em um disco colorido que incorpora as cores primárias do espectro visível, a saber, vermelho, laranja, amarelo, verde, azul, anil e violeta.

A cor é uma característica física dos objetos, relacionada às suas propriedades de refração em diferentes comprimentos de onda \citep{Finlay2004}. Cada comprimento corresponde a uma cor específica do espectro cromático, que é a decomposição da luz visível. Esse espectro abrange tons que vão do vermelho ao azul violeta, com comprimentos além do vermelho sendo infravermelhos e abaixo do azul violeta ultravioleta, ambos invisíveis. Os comprimentos de onda refletidos pelos objetos entram nos nossos olhos e são interpretados pela retina, que forma a imagem que percebemos.
Quando o disco é colocado em rotação a altas velocidades, observa - se a composição da luz branca.

Em estado estacionário, a separação das cores é claramente perceptível; no entanto, ao girar o disco, as cores se misturam, resultando na aparência de um disco branco. Newton utilizou este
dispositivo para evidenciar que a combinação das cores visíveis resulta na cor branca.

A decomposição da luz observada no disco de Newton está intrinsecamente ligada ao experimento realizado por Isaac Newton com um prisma. Newton evidenciou que a luz branca é constituída por diversas cores que, ao serem refratadas pelo prisma, formam o espectro visível. No disco de Newton, o fenômeno ocorre de maneira inversa: ao girar rapidamente o disco que contém as cores do arco-íris, as cores se amalgamam e nossos olhos percebem novamente a luz branca, demonstrando que a combinação das cores pode restituir a luz original.

Adicionalmente, é fundamental distinguir entre dois tipos de mistura de cores. Na mistura aditiva, que se dá com fontes de luz, as cores primárias (vermelho, verde e azul) se somam para gerar novas tonalidades, resultando na luz branca quando combinadas em sua totalidade. Na mistura subtrativa, que ocorre com pigmentos ou tintas, as cores primárias (ciano, magenta e amarelo) absorvem parte da luz; ao serem misturadas, tendem a produzir uma tonalidade próxima ao preto, uma vez que se dá a subtração das componentes luminosas refletidas.

\section{Objetivos}\label{objetivos-2}

\begin{itemize}
\item
  Visualização da composição da luz em razão do fenômeno da persistência da luz
\item
  Compreender o fenômeno de decomposição da luz branca e sua recomposição através das sete cores do espectro visível (vermelho,laranja,amarelo,verde,azul,anil e violeta). E reconhecer que a luz branca resulta da soma das cores que compõem o espectro.
\item
  Relacionar o experimento ao estudo da óptica e física da luz.
\end{itemize}

-Relacionar a experiência ao conceito de síntese aditiva de cores da luz.

\section{Parte experimental}\label{parte-experimental-2}

\subsection{Materiais utilizados}\label{materiais-utilizados-3}

Uma cartolina branca ou papelão, marcadores ou tinta com as cores vermelho, laranja, amarelo, verde, azul, anil e violeta,cola branca, tesoura,régua,compasso,lápis,furador de papel,fita de qualquer tipo (adesiva, isolante ou de vedação), uma base de MDF 26 × 16, uma placa de MDF 16×6, 4 parafusos, bateria 24 V, motor 12 V, serra circular,dois fios de cobre, serra copo e parafusadeira.

\subsection{Procedimento experimental}\label{procedimento-experimental-2}

\subsubsection{Preparação da base}\label{preparauxe7uxe3o-da-base}

\begin{enumerate}
\def\labelenumi{\arabic{enumi}.}
\item
  Corte um MDF 26 × 16. Com o serra copo,faça um círculo no centro da madeira.
\item
  Acople o motor de 12v.
\item
  Insira uma das pontas do fio de cobre na parte positiva do motor e a outra ponta no terminal na parte negativa.
\item
  Corte um novo MDF 16×6 para servir como base, parafuse essa madeira menor na parte inferior da madeira maior.
\end{enumerate}

\subsubsection{Preparação do disco}\label{preparauxe7uxe3o-do-disco}

\begin{enumerate}
\def\labelenumi{\arabic{enumi}.}
\item
  Com o compasso, faça um círculo na folha de cartolina e um círculo no papelão . O tamanho aqui não é importante, porém, se o disco for muito grande, o efeito da composição da luz branca será menos visível;
\item
  Com a régua e o lápis, faça divisões triangulares no círculo da folha de cartolina, todas com o mesmo tamanho, como uma pizza.
\item
  Pinte os triângulos e certifique-se de todo o espaço estar totalmente colorido e sem falhas; 4. Faça um pequeno furo circular no centro dos círculos com o furador de papel.Passe cola branca e faça um apoio com um novo circulo de papelão.
\item
  Acople esse disco na haste do motor de 12V. Na sequência conecte uma das pontas do fio de cobre ao polo positivo e a outra ponta ao polo negativo da bateria de 24V.\\
\item
  Finalmente é só garantir a rotação do disco. Quanto mais rápido o disco girar, maior será o efeito da composição da luz branca sobre os nossos olhos.
\end{enumerate}

\begin{figure}

{\centering \includegraphics[width=0.8\linewidth]{img/05/figura-5a} 

}

\caption{Disco de Newton estático, com as sete cores separadas utilizado para demonstrar a composição da luz branca.}\label{fig:figura-5a}
\end{figure}

\begin{figure}

{\centering \includegraphics[width=0.8\linewidth]{img/05/figura-5b} 

}

\caption{Montagem do disco de Newton.}\label{fig:figura-5b}
\end{figure}

\begin{figure}

{\centering \includegraphics[width=0.8\linewidth]{img/05/figura-5c} 

}

\caption{Montagem experimental do Disco de Newton conectado a um motor elétrico de 12 volts, alimentado por uma bateria de 24 volts, permitindo alta rotação para demonstrar a mistura das cores e a formação da cor branca.}\label{fig:figura-5c}
\end{figure}

\begin{figure}

{\centering \includegraphics[width=0.8\linewidth]{img/05/figura-5d} 

}

\caption{Disco de Newton em movimento.Ao girar, as cores se misturam visualmente, resultando na percepção de cor branca.}\label{fig:figura-5d}
\end{figure}

\section{Questionário}\label{questionuxe1rio-4}

\begin{enumerate}
\def\labelenumi{\arabic{enumi}.}
\item
  O que é o Disco de Newton ? A luz branca é formada por quais cores?
\item
  O que se observa quando o disco está parado? E quando o disco está girando rapidamente?
\item
  Explique por que o disco parece ficar branco quando gira, mesmo estando pintado com várias cores.
\item
  Qual a diferença entre a mistura de cores de luz (adição) e a mistura de cores de tinta (subtração)?
\item
  Como esse experimento pode ser relacionado ao arco-íris?
\item
  Se o disco de Newton fosse pintado com cores desiguais (por exemplo, mais vermelho e menos azul), o que mudaria na observação ao girar?
\item
  Explique, com suas palavras, a importância desse experimento para a compreensão da natureza da luz.
\end{enumerate}

\chapter{Capacitor caseiro Garrafa de Leiden}\label{capacitor-caseiro-garrafa-de-leiden}

\section{Autora: Sabrina Vitoria Wersel}\label{autora-sabrina-vitoria-wersel}

\section{Introdução}\label{introduuxe7uxe3o-6}

A garrafa de Leiden é um capacitor caseiro que pode ser facilmente construído com materiais simples, permitindo armazenar energia elétrica. Consiste em um recipiente isolante (vidro ou plástico) com um condutor interno e um externo. Para usá-la, conecta-se o condutor interno a uma fonte eletrostática para acumular cargas elétricas; quando o contato é feito com a camada externa, a energia é liberada, produzindo uma descarga, ou seja, uma faísca.

A bilha de Leiden, também conhecida como Garrafa de Leiden, é um dispositivo histórico que desempenhou um papel importante no desenvolvimento da eletricidade. Inventada no século 18 pelo físico Pieter Van Musschenbroek, esta garrafa consiste em um recipiente de vidro com revestimento interno e externo de metal, separados por um isolante. Seu descobrimento revolucionou o campo da eletrostática e ajudou a avançar no estudo da eletricidade.

\section{Objetivos}\label{objetivos-3}

\begin{itemize}
\item
  Explicar o funcionamento da garrafa de Leiden de forma claraa e acessível, como a garrafa acumula carga elétrica e como isso está relacionado ao princípio da eletricidade estática;
\item
  Despertar o interesse dos alunos pelo estudo da eletricidade, através de atividades práticas e interativas;
\item
  Explorar as aplicações históricas da Garrafa de Leiden, em como auxíliou em descobertas científicas;
\end{itemize}

\section{Parte experimental}\label{parte-experimental-3}

\subsection{Materiais utilizados}\label{materiais-utilizados-4}

\begin{itemize}
\item
  Um recipiente: Pode ser uma garrafa de vidro, pote de plástico ou caixa de comprimidos.
\item
  Papel alumínio
\item
  Palito de madeira
\item
  Cola quente
\item
  Bola de pingue-pongue
\item
  Fio de cobre
\item
  Fita isolante
\end{itemize}

\subsection{Procedimento experimental}\label{procedimento-experimental-3}

\begin{enumerate}
\def\labelenumi{\alph{enumi}.}
\item
  revestir o pote de plástico com papel alumínio por dentro e por fora;
\item
  Cortar o palito de madeira até o tamanho do recipiente usado
\item
  fazer um furo na tampa do pote e na bolinha de pingue pongue;
\item
  fixar o palito de madeira dentro do furo da bolinha, em seguida revestir a bolinha com papel alumínio e colar uma das extremidades do fio de cobre no palito de modo que encoste no papel alumínio da bolinha de pingue-pongue;
\item
  atravessar a outra extremidade do palito dentro da tampa do recipiente;
\item
  fixar a outra extremidade do fio de cobre no papel alumínio de dentro do pote;
\item
  atritar um cano de pvc com papel, (ou uma bexiga de plástico sobre os cabelos) em seguida encostar o material atritado sobre a esfera de alumínio para a passagem de cargas.
\end{enumerate}

\section{Questionário}\label{questionuxe1rio-5}

\begin{enumerate}
\def\labelenumi{\arabic{enumi}.}
\item
  Por que a garrafa de leiden consegue armazenar energia?
\item
  Explique porque levamos um choque quando tocamos na esfera de alumínio da garrafa de leiden.
\item
  Em termos de eletricidade, o que acontece quando atritamos o cano de pvc com um papel toalha?
\end{enumerate}

\chapter{Experimento de Refração de luz: Seta invertida}\label{experimento-de-refrauxe7uxe3o-de-luz-seta-invertida}

\section{Autor: Vinicius Alady Jankovski}\label{autor-vinicius-alady-jankovski}

\section{Introdução}\label{introduuxe7uxe3o-7}

A luz é um dos fenômenos naturais mais fascinantes e essenciais para a nossa percepção do mundo. Entre os diversos comportamentos que ela apresenta, a refração se destaca por sua capacidade de alterar a trajetória da luz ao atravessar diferentes meios, como o ar, a água ou o vidro. Esse fenômeno está presente em diversas situações do cotidiano --- desde o aparente ``quebra'' de um lápis dentro de um copo com água até o funcionamento de lentes em óculos, câmeras e microscópios.

Nesta apostila, exploraremos a refração de forma prática e visual por meio do experimento conhecido como Seta Invertida. Com materiais simples e acessíveis, será possível observar como a luz pode criar ilusões ópticas surpreendentes, invertendo a direção de uma imagem ao passar por um copo com água. Além de despertar a curiosidade, esse experimento ajuda a compreender conceitos fundamentais da óptica e estimula o pensamento científico por meio da observação e da análise.

\section{Objetivos}\label{objetivos-4}

\begin{itemize}
\item
  Demonstrar como a refração da luz pode causar a inversão aparente de uma imagem ao passar por diferentes meios (vidro e água).
\item
  Identificar como a refração pode alterar a trajetória da luz e a percepção visual de objetos.
\item
  Demonstrar, por meio de um experimento simples, como a refração pode gerar ilusões de ótica.
\item
  Relacionar o comportamento da luz com aplicações práticas em lentes, óculos e instrumentos ópticos.
\item
  Aplicar conceitos da física óptica para explicar fenômenos visuais do cotidiano.
\end{itemize}

\section{Parte experimental}\label{parte-experimental-4}

\subsection{Materiais necessários}\label{materiais-necessuxe1rios-1}

\begin{itemize}
\tightlist
\item
  1 copo de vidro transparente (cilíndrico)
\item
  Água
\item
  1 folha de papel
\item
  Caneta ou marcador
\item
  Fita adesiva (opcional)
\end{itemize}

\subsection{Procedimento experimental}\label{procedimento-experimental-4}

\begin{enumerate}
\def\labelenumi{\arabic{enumi}.}
\item
  Desenhe a seta Na folha de papel, desenhe uma seta horizontal grande (→). Você pode fazer várias setas apontando para o mesmo lado para reforçar o efeito.
\item
  Fixe o papel Cole o papel na parede ou em uma superfície vertical, na altura dos olhos. Certifique-se de que a seta esteja visível através do copo.
\item
  Posicione o copo vazio Coloque o copo vazio entre seus olhos e o papel com a seta. Observe a seta através do copo --- ela deve parecer normal.
\item
  Adicione água ao copo Encha o copo com água até o topo. Agora olhe novamente através do copo.
\end{enumerate}

\subsection{\texorpdfstring{\textbf{Observação}}{Observação}}\label{observauxe7uxe3o}

A seta aparecerá invertida, apontando para o lado oposto (←). Isso acontece por causa da refração da luz --- a luz que passa pelo vidro e pela água é desviada, criando uma imagem invertida.

\subsection{\texorpdfstring{\textbf{Explicação científica}}{Explicação científica}}\label{explicauxe7uxe3o-cientuxedfica}

\begin{itemize}
\item
  A luz viaja em linha reta, mas muda de direção ao passar de um meio para outro (como do ar para a água).
\item
  O copo cilíndrico com água atua como uma lente convergente, que pode inverter a imagem dependendo da curvatura e da posição do observador.
\item
  Esse fenômeno é um exemplo claro de refração, que ocorre quando a luz muda de velocidade ao atravessar diferentes materiais.
\end{itemize}

\section{Atividades complementares}\label{atividades-complementares}

\begin{itemize}
\item
  Teste com diferentes formatos de copo (quadrado, cônico).
\item
  Use outros desenhos além da seta (palavras, números).
\item
  Meça a distância em que a inversão ocorre com maior nitidez.
\end{itemize}

\section{Aplicações do Fenômeno}\label{aplicauxe7uxf5es-do-fenuxf4meno}

\begin{itemize}
\item
  Lentes de óculos e câmeras
\item
  Microscópios e telescópios
\item
  Óptica médica (como lentes intraoculares)
\item
  Ilusões ópticas e arte visual
\end{itemize}

\section{Conclusão}\label{conclusuxe3o-1}

Esse experimento simples mostra como a refração pode alterar nossa percepção visual. É uma ótima forma de introduzir conceitos de óptica e física de maneira prática e divertida.

\section{Questionário}\label{questionuxe1rio-6}

\begin{enumerate}
\def\labelenumi{\arabic{enumi}.}
\tightlist
\item
  O que é refração da luz?
\end{enumerate}

Resposta: Refração é o fenômeno físico que ocorre quando a luz muda de velocidade ao passar de um meio para outro, como do ar para a água, fazendo com que ela mude de direção.

\begin{enumerate}
\def\labelenumi{\arabic{enumi}.}
\setcounter{enumi}{1}
\tightlist
\item
  Por que a seta parece invertida quando observada através do copo com água?
\end{enumerate}

Resposta: A imagem da seta se inverte devido à refração da luz pelas superfícies curvas do copo e pela água. A luz é desviada de forma que a imagem parece estar virada para o lado oposto.

\begin{enumerate}
\def\labelenumi{\arabic{enumi}.}
\setcounter{enumi}{2}
\tightlist
\item
  Esse experimento é um exemplo de qual tipo de ilusão de ótica?
\end{enumerate}

Resposta: É um exemplo de ilusão de ótica causada pela refração, onde a imagem percebida é diferente da real devido à mudança na trajetória da luz.

\begin{enumerate}
\def\labelenumi{\arabic{enumi}.}
\setcounter{enumi}{3}
\tightlist
\item
  Cite uma aplicação prática da refração da luz no cotidian
\end{enumerate}

Resposta: A refração é usada em lentes de óculos, câmeras fotográficas, microscópios, telescópios e instrumentos ópticos em geral para corrigir ou ampliar imagens.

\bibliography{bibliografia.bib}

\end{document}
